\documentclass[12pt]{article}
\usepackage{polski}
\usepackage[utf8]{inputenc}
\usepackage{amssymb}
\usepackage{amsfonts}
\usepackage{stmaryrd}
\usepackage{amsmath}
\usepackage{fancyvrb} 
\usepackage{graphicx}
\usepackage{psfrag}
\usepackage{wrapfig}

\usepackage[a4paper,left=2.3cm,right=2.3cm,top=1.3cm,bottom=3cm]{geometry}
\sloppy

\title{Kolokwium II\\grupa A}
\date{22 stycznia 2010}

\begin{document}
\maketitle
\DefineShortVerb{\|}
\thispagestyle{empty}
\begin{enumerate}


\item \textbf{(6 pkt)}
Napisz makrodefinicję preprocesora,
której zastosowanie z~nazwą dowolnego
typu jako argumentem spowoduje wypisanie
nazwy typu a~po niej rozmiaru typu (w~bajtach).
Przykładowe użycie: |PRINT_SIZE(char)|
powinno spowodować wypisanie na~ekranie np.:
,,Typ char zajmuje 1B''.

\item \textbf{(6 pkt)}
Napisz funkcję |double abs_diff(const double *a, const double *b)|,
która zwraca wartość bezwzględną różnicy
wartości wskazywanych przez parametry wejściowe |a| i |b|.

\item \textbf{(11 pkt)}
Napisz funkcję, która dostaje jako argumenty
dwie dwuwymiarowe tablice tablic o~elementach
typu char oraz ich wymiary
i~zwraca jako wartość 1
jeżeli elementy obu tablic tworzą identyczne
zbiory wartości (wartość $x$ występuje
w pierwszej tablicy wtedy i~tylko wtedy
gdy występuje w~drugiej)
i~0 w~przeciwnym wypadku.

\item \textbf{(9 pkt)}
Napisz funkcję, która otrzymuje jako argument
napis, będący ścieżką do~pliku
i~zwraca liczbę słów
(ciągów znaków rozdzielonych białymi znakami)
zawartych w~pliku o~podanej nazwie.

\item \textbf{(8 pkt)}
Napisz funkcję, która otrzymuje jako argumenty
dwa napisy i~przepisuje pierwszy do~drugiego od~końca.
Zakładamy, że~pierwszy napis zmieści się w~drugim.
Nie jest dozwolone użycie funkcji bibliotecznych.

\item \textbf{(10 pkt)}
Napisz funkcję, która porównuje dwie listy
bez~głowy o~elementach typu:
\begin{verbatim}
struct element {
  int x;
  struct element * next;
};
\end{verbatim}
i~zwraca 1 jeżeli obie listy są równe
(odpowiadające sobie elementy mają
te same wartości pola |x|)
oraz 0 w~przeciwnym razie.

\end{enumerate}

\vspace{1cm}
\textbf{Uwagi}
\begin{itemize}
  \item W każdym zadaniu należy dopisać tylko te nagłówki bibliotek, z których korzystamy.
  \item Zakładamy, że dane wejściowe spełniają określone w treści zadania warunki,
        więc nie~trzeba sprawdzać ich poprawności.
  \item Prace nieczytelne nie będą sprawdzane.
\end{itemize}

\end{document}
