\documentclass[12pt]{article}
\usepackage{polski}
\usepackage[utf8]{inputenc}
\usepackage{amssymb}
\usepackage{amsfonts}
\usepackage{stmaryrd}
\usepackage{amsmath}
\usepackage{fancyvrb} 
\usepackage[top=2.5cm, bottom=1.5cm, left=2.5cm, right=2.5cm]{geometry}

\pagestyle{empty}
\title{Kolokwium}

\begin{document}
\DefineShortVerb{\@}
\begin{center}
\Large{Podstawy programowania}\\
\large{Kolokwium I --- grupa A}\\
\small{25 listopada 2011}
\end{center}

\begin{enumerate}
\item \textbf{(5 pkt)} Napisz program, który wczytuje dwie liczby całkowite $x$ i $y$, i~zwraca jako wartość~$\frac{|x|+|y|}{2}$.
\item \textbf{(6 pkt)} Napisz program, który pobierze od~użytkownika dwie liczby całkowite $n$ i~$m$
	(zakładamy, że~$n < m$). Program ma wyświetlić wszytkie liczby z~przedziału domkniętego $[n; m]$,
	które są podzielne przez~3 lub podzielne przez~5.
\item \textbf{(7 pkt)} Napisz program, który dopóty będzie wczytywał ze~standardowego wejścia liczby naturalne,
	dopóki ich suma nie~przekroczy liczby 100. Program na~standardowym wyjściu powinien
	wyświetlic wartość o~jaką obliczona suma przekroczyła liczbę~100.

\item  \textbf{(7 pkt)}
Napisz rekurencyjną funkcję, która przyjmuje liczbę całkowitą $n$ i~zwraca wartość $n$-tego wyrazu ciągu zdefiniowanego wzorem:
$$
\begin{cases}
	a_0 & = 0,\\
	a_1 & = 1,\\
	a_{2n} & = a_{2n-1} + a_{2n-2},\\
	a_{2n+1} & = a_{2n} - a_{2n-1}.
\end{cases}
$$

\item \textbf{(6 pkt)}
Napisz funkcję, która na~wejściu otrzyma jedną tablicę z~wartościami całkowitymi
i~jej rozmiar a~na wyjściu zwróci liczbę elementów których cyfra jedności wynosi~3.
Przykład:\\
we: @2 14 3 26 5 34 33 13 65@\\
wy: @3@

\item \textbf{(9 pkt)}
Napisz funkcję, która na~wejściu otrzyma 2 tablice z~wartościami rzeczywistymi i~ich rozmiar.
Funkcja ma wymienić między tablicami elementy o~jednakowych indeksach tak, aby element większy znalazł się w~tablicy pierwszej.
Przykład:\\
\vspace{-0.5cm}
\begin{center}
\begin{tabular}{r | c | c}
		& przed zamianą 	& po zamianie\\ \hline
tablica 1	& @1 5 -1 12  6@ 	& @2 5 -1 12 10@\\
tablica 2	& @2 4 -6  8 10@ 	& @1 4 -6  8  6@
\end{tabular}
\end{center}
\end{enumerate}

\vfill

\textbf{Uwagi}
\begin{itemize}
 \item W każdym zadaniu (także w tych, w których trzeba napisać tylko funkcję) należy dopisać nagłówki bibliotek, z których korzystamy.
 \item Zakładamy, że dane wejściowe spełniają określone w treści zadania warunki, więc nie trzeba sprawdzać ich poprawności.
 \item Prace nieczytelne nie będą sprawdzane.
\end{itemize}

\newpage

\begin{center}
\Large{Podstawy programowania}\\
\large{Kolokwium I --- grupa B}\\
{25 listopada 2011}
\end{center}

\begin{enumerate}
\item \textbf{(5 pkt)} Napisz program, który wczytuje dwie liczby całkowite $x$ i $y$, i~zwraca jako wartość~$\frac{|x|+|y|}{2}$.
\item \textbf{(6 pkt)} Napisz program, który pobierze od~użytkownika dwie liczby całkowite $n$ i~$m$
	(zakładamy, że~$n < m$). Program ma wyświetlić wszytkie liczby z~przedziału domkniętego $[n; m]$,
	które są podzielne przez~3 lub podzielne przez~5.
\item \textbf{(7 pkt)} Napisz program, który dopóty będzie wczytywał ze~standardowego wejścia liczby naturalne,
	dopóki ich suma nie~przekroczy liczby 100. Program na~standardowym wyjściu powinien
	wyświetlic wartość o~jaką obliczona suma przekroczyła liczbę~100.

\item  \textbf{(7 pkt)}
Napisz rekurencyjną funkcję, która przyjmuje liczbę całkowitą $n$ i~zwraca wartość $n$-tego wyrazu ciągu zdefiniowanego wzorem:
$$
\begin{cases}
	a_0 & = 0,\\
	a_1 & = 1,\\
	a_{2n} & = a_{2n-1} + a_{2n-2},\\
	a_{2n+1} & = a_{2n} - a_{2n-1}.
\end{cases}
$$

\item \textbf{(6 pkt)}
Napisz funkcję, która na~wejściu otrzyma jedną tablicę z~wartościami całkowitymi
i~jej rozmiar a~na wyjściu zwróci liczbę elementów których cyfra jedności wynosi~3.
Przykład:\\
we: @2 14 3 26 5 34 33 13 65@\\
wy: @3@

\item \textbf{(9 pkt)}
Napisz funkcję, która na~wejściu otrzyma 2 tablice z~wartościami rzeczywistymi i~ich rozmiar.
Funkcja ma wymienić między tablicami elementy o~jednakowych indeksach tak, aby element większy znalazł się w~tablicy pierwszej.
Przykład:\\
\vspace{-0.5cm}
\begin{center}
\begin{tabular}{r | c | c}
		& przed zamianą 	& po zamianie\\ \hline
tablica 1	& @1 5 -1 12  6@ 	& @2 5 -1 12 10@\\
tablica 2	& @2 4 -6  8 10@ 	& @1 4 -6  8  6@
\end{tabular}
\end{center}
\end{enumerate}

\vfill

\textbf{Uwagi}
\begin{itemize}
 \item W każdym zadaniu (także w tych, w których trzeba napisać tylko funkcję) należy dopisać nagłówki bibliotek, z których korzystamy.
 \item Zakładamy, że dane wejściowe spełniają określone w treści zadania warunki, więc nie trzeba sprawdzać ich poprawności.
 \item Prace nieczytelne nie będą sprawdzane.

\end{itemize}

\end{document}

