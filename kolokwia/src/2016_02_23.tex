\documentclass[extrafontsizes,12pt]{article}
\usepackage{polski}
\usepackage[utf8]{inputenc}
\usepackage{amssymb}
\usepackage{amsfonts}
\usepackage{stmaryrd}
\usepackage{amsmath}
\usepackage{fancyvrb}
\usepackage{graphicx}
\usepackage{psfrag}
\usepackage{wrapfig}

\usepackage[a4paper,left=1.5cm,right=1.5cm,top=2cm,bottom=1cm]{geometry}
\sloppy

\title{Podstawy Programowania --- kolokwium poprawkowe}
\date{23 lutego 2016}

\begin{document}

\maketitle
\DefineShortVerb{\|}
\thispagestyle{empty}

\begin{enumerate}
\itemsep4pt

\item \textbf{(20 pkt)}
Napisz funkcję, która dostaje jako parametr dynamiczną
dwuwymiarową tablicę liczb całkowitych i~jej wymiary $n$, $m$.
Funkcja ma zwrócić~1, jeśli na~brzegach
(tzn.~w~pierwszym i~ostatnim wierszu oraz~w~pierwszej
i~ostatniej kolumnie) tablicy występują wartości tylko niezerowe;
w~przeciwnym razie funkcja ma zwrócić~0. 

\item \textbf{(20 pkt)}
Napisz program składający się z~funkcji |main|
oraz~co~najmniej jednej funkcji pomocniczej.
Program w~funkcji main ma wczytać od~użytkownika liczbę całkowitą~$n$ oraz~$x$
a~następnie zaalokować dynamiczną tablicę liczb całkowitych
o~rozmiarze~$n$ elementów i~wczytać do~niej $n$~wartości.
Następnie ma być wywołana funkcja,
która przyjmuje jako parametry liczbę~$n$, $x$
oraz~wypełnioną już wartościami wspomnianą wcześniej tablicę.
Funkcja ma zwrócić sumę kwadratów liczb z~tablicy,
które nie~są wielokrotnościami liczby~$x$.
Zwrócona wartość ma być wypisana przez~funkcję |main| na~standardowe wyjście.
Nie~zapomnij zwolnić pamięci zajmowanej przez~tablicę.

\item \textbf{(30 pkt)}
Zdefiniuj strukturę planeta, posiadającą:
numer planety (liczba całkowita),
nazwa planety (napis 50-literowy),
promień planety (liczba zmiennoprzecinkowa).
Napisz funkcję przyjmującą tablicę planet
oraz~jej rozmiar.
Funkcja ma zwrócić planetę (całą strukturę)
o~największej objętości ($V = \frac{4}{3} \cdot \pi \cdot r^3$).
Okazuje się jednak, że~planety o~numerach nieparzystych
nie są idealnymi kulami, dlatego w~ich przypadku
objętość należy pomniejszyć o~10\%
w~stosunku do~wyniku uzyskanego ze~wzoru.

\item \textbf{(30 pkt)}
Napisz funkcję przyjmującą napis i~zwracającą liczbę par znaków
(parą nazywamy dwa sąsiadujące jednakowe znaki lub~litery
bez~względu na~ich wielkość).
Przykład:
\begin{center}
\begin{tabular}{ r c l }
wejście & & wyjście \\
\hline
1337 & $\rightarrow$ & 1 \\
bbaaac & $\rightarrow$ & 3 \\
AabBb & $\rightarrow$ & 3 \\
\end{tabular}
\end{center}

\end{enumerate}

\vfill

\textbf{Uwagi}

\begin{itemize}
\item W~każdym zadaniu (także w~tych, w~których trzeba napisać tylko funkcję) należy dopisać pliki nagłówkowe, z~których korzystamy.
\item Rozwiązanie każdego zadania może zawierać dowolną liczbę funkcji pomocniczych.
\item Zakładamy, że dane wejściowe spełniają określone w treści zadania warunki, więc nie~trzeba sprawdzać ich poprawności.
\item Prace nieczytelne nie będą sprawdzane.
\item Każde zadanie należy rozwiązać na~osobnej, podpisanej kartce. Wszystkie kartki (nawet puste) należy oddać.
\end{itemize}
\end{document}
