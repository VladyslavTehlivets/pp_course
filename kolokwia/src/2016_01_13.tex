\documentclass[extrafontsizes,10pt]{article}
\usepackage{polski}
\usepackage[utf8]{inputenc}
\usepackage{amssymb}
\usepackage{amsfonts}
\usepackage{stmaryrd}
\usepackage{amsmath}
\usepackage{fancyvrb}
\usepackage{graphicx}
\usepackage{psfrag}
\usepackage{wrapfig}

\usepackage[a4paper,left=2cm,right=2cm,top=1.5cm,bottom=2cm]{geometry}
\sloppy

\title{Podstawy Programowania --- kolokwium II}
\date{13 stycznia 2016}

\begin{document}

\maketitle
\DefineShortVerb{\|}
\thispagestyle{empty}

\begin{enumerate}
\itemsep1em

\item \textbf{(5 pkt)}
Napisz makrodefinicję, która sprawdzi,
czy podana w~parametrze wartość liczbowa
jest podzielna przez~10
i~zwróci wartość~1 jeśli warunek podzielności jest spełniony
lub~0 w~przeciwnym wypadku.

\item \textbf{(10 pkt)}
Napisz funkcję, która jako parametr przyjmie liczbę całkowitą~$n$.
Funkcja ma zaalokować dynamiczną dwuwymiarową trójkątną tablicę liczb całkowitych,
wypełnić ją zerami i~zwrócić jej adres.
Na~przykład dla~$n = 5$ tablica powinna wyglądać następująco:
\begin{center}
\begin{BVerbatim}
0
0 0
0 0 0
0 0 0 0
0 0 0 0 0
\end{BVerbatim}
\end{center}
Liczba wierszy w trójkątnej tablicy równa jest~$n$,
natomiast liczba kolumn w~każdym wierszu jest o~1
większa od~liczby kolumn w~wierszu poprzednim.
Wiersz ostatni ma liczbę kolumn równą liczbie wierszy, czyli~$n$.

\item \textbf{(10 pkt)}
Napisz funkcję, która przyjmie w~parametrze jednowymiarową tablicę
liczb typu |int| o~rozmiarze $n$ oraz~liczbę całkowitą~$n$.
Funkcja ma zaalokować tablicę i~zwrócić jej~adres,
uprzednio zapisując w~niej odwrotności liczb z~przekazanej
w~parametrze tablicy.
Odwrotność liczby~$x$ to liczba~$\frac{1}{x}$.
Zakładamy, że~odwrotnością liczby~0 jest liczba~0.

\item \textbf{(10 pkt)}
Napisz funkcję, która przyjmuje rozmiar~$n$
i~trzy tablice jednowymiarowe
liczb całkowitych
(dwie pierwsze mają rozmiar~$n$,
a~ostatnia rozmiar~$2\cdot n$).
Funkcja ma podstawić wartości
z~dwóch pierwszych tablic
do~trzeciej tablicy według reguły:
|tab1 = {3, 3, 3} tab2 = {4, 4, 4}| $\rightarrow$ |tab3 = {3, 4, 3, 4, 3, 4}|.

\item \textbf{(15 pkt)}
Napisz funkcję, która otrzymuje w~parametrach
dynamicznie alokowaną dwuwymiarową tablicę
liczb całkowitych oraz~liczbę jej wierszy.
Wiersze tej~tablicy są różnej długości,
jednak wiadomo, że~każdy z~nich kończy się liczbą~0.
Funkcja powinna z~każdego wiersza usunąć liczby
będące wielokrotnościami liczby znajdującej się
w~pierwszym polu tego wiersza (łącznie z~nią samą)
pozostawiając~0 na~końcu.
Jeżeli wiersz zaczyna się liczbą~0,
pozostawiany jest bez~zmian.
Kolejność liczb w~wierszu
po~tej operacji jest dowolna.
\begin{center}
\begin{tabular}{ l l }
wejście & wyjście \\
\hline
3 5 6 8 9 0 & 5 8 0 \\
1 4 6 0 & 0 \\
0 & 0 \\
2 3 4 5 6 7 8 0 & 3 5 7 0
\end{tabular}
\end{center}


\end{enumerate}

\vfill

\textbf{Uwagi}

\begin{itemize}
\item W~każdym zadaniu (także w~tych, w~których trzeba napisać tylko funkcję) należy dopisać pliki nagłówkowe, z~których korzystamy.
\item Rozwiązanie każdego zadania może zawierać dowolną liczbę funkcji pomocniczych.
\item Zakładamy, że dane wejściowe spełniają określone w treści zadania warunki, więc nie~trzeba sprawdzać ich poprawności.
\item Prace nieczytelne nie będą sprawdzane.
\item Każde zadanie należy rozwiązać na~osobnej, podpisanej kartce. Wszystkie kartki (nawet puste) należy oddać.
\end{itemize}
\end{document}
